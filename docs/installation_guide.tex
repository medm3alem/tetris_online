\documentclass[11pt,a4paper]{article}

% ========================
% Packages
% ========================
\usepackage[utf8]{inputenc}
\usepackage[T1]{fontenc}
\usepackage[french]{babel}
\usepackage{geometry}
\usepackage{hyperref}
\usepackage{listings}
\usepackage{xcolor}
\usepackage{titlesec}
\usepackage{graphicx}

\geometry{margin=2.5cm}

% ========================
% Code style
% ========================
\definecolor{codegray}{rgb}{0.95,0.95,0.95}

\lstset{
    backgroundcolor=\color{codegray},
    basicstyle=\ttfamily\small,
    breaklines=true,
    frame=single,
    language=bash
}

% ========================
% Title formatting
% ========================
\titleformat{\section}{\large\bfseries}{\thesection}{1em}{}
\titleformat{\subsection}{\normalsize\bfseries}{\thesubsection}{1em}{}

% ========================
% Document
% ========================
\begin{document}

% ========================
% Page de garde
% ========================
\begin{titlepage}
    \centering

    % Logo ENSTA
    \includegraphics[width=4cm]{ensta.png}
    \vspace{1.5cm}

    {\Large\bfseries ENSTA Paris\par}
    \vspace{0.5cm}

    {\large\bfseries Matière : IN204\par}
    \vspace{2cm}

    {\Huge\bfseries Guide d'installation et d'exécution\par}
    \vspace{0.4cm}
    {\Huge\bfseries du projet Tetris en ligne\par}

    \vspace{1.5cm}
    {\Large Windows + WSL (Ubuntu) / Linux natif\par}

    \vspace{2.5cm}

    {\large\bfseries Réalisé par :\par}
    \vspace{0.3cm}
    {\large ELADEB Mohamed\par}
    {\large ABID Mahdi\par}

    \vfill

    {\large Année universitaire 2025 -- 2026\par}

\end{titlepage}

\tableofcontents
\newpage

% ========================
\section{Installation de WSL}

Sous Windows, ouvrir \textbf{PowerShell en administrateur} et exécuter :

\begin{lstlisting}
wsl --install
\end{lstlisting}

Redémarrer le PC lorsque cela est demandé, puis lancer Ubuntu :

\begin{lstlisting}
wsl
\end{lstlisting}

% ========================
\section{Installation des outils de base}

Dans le terminal WSL :

\begin{lstlisting}
sudo apt update
sudo apt install -y build-essential cmake git g++ net-tools
\end{lstlisting}

% ========================
\section{Installation des dépendances graphiques (Raylib)}

\begin{lstlisting}
sudo apt install -y \
libx11-dev \
libxrandr-dev \
libxinerama-dev \
libxcursor-dev \
libxi-dev \
libgl1-mesa-dev
\end{lstlisting}

% ========================
\section{Installation de Raylib}

\begin{lstlisting}
cd ~
git clone https://github.com/raysan5/raylib.git
cd raylib
mkdir build && cd build
cmake ..
make -j4
sudo make install
sudo ldconfig
\end{lstlisting}

% ========================
\section{Installation du support audio (optionnel)}

\begin{lstlisting}
sudo apt install -y pulseaudio pulseaudio-utils alsa-utils
pulseaudio --start
\end{lstlisting}

Test du son :

\begin{lstlisting}
speaker-test -t sine -f 440
\end{lstlisting}

% ========================
\section{Récupération du projet}

\textbf{Important :} ne pas cloner le projet dans \texttt{/mnt/c}.

\begin{lstlisting}
cd ~
git clone https://github.com/medm3alem/tetris_online.git
\end{lstlisting}

% ========================
\section{Compilation du serveur}

\begin{lstlisting}
cd ~/tetris_online/server
g++ -std=c++17 server.cpp -o server -lpthread
\end{lstlisting}

% ========================
\section{Compilation du client}

\begin{lstlisting}
cd ~/tetris_online/client
mkdir build
cd build
cmake ..
make
\end{lstlisting}

% ========================
\section{Lancement du serveur}

\begin{lstlisting}
cd ~/tetris_online/server
./server
\end{lstlisting}

Le serveur écoute sur le port \textbf{4242}.

% ========================
\section{Configuration réseau Windows ↔ WSL}

\subsection{Récupération des adresses IP}

Adresse IP Windows :

\begin{lstlisting}
ipconfig
\end{lstlisting}

Adresse IP WSL :

\begin{lstlisting}
hostname -I
\end{lstlisting}

\subsection{Redirection de port}

Dans \textbf{PowerShell en administrateur} :

\begin{lstlisting}
netsh interface portproxy add v4tov4 ^
listenport=4242 listenaddress=0.0.0.0 ^
connectport=4242 connectaddress=IP_WSL
\end{lstlisting}

\subsection{Ouverture du port dans le pare-feu}

\begin{lstlisting}
New-NetFirewallRule ^
-DisplayName "Tetris WSL Server" ^
-Direction Inbound ^
-Protocol TCP ^
-LocalPort 4242 ^
-Action Allow ^
-Profile Any
\end{lstlisting}

% ========================
\section{Lancement du client}

Avant de lancer le client, l'utilisateur doit récupérer l'adresse IP
de la machine sur laquelle le serveur est exécuté.

Sous Windows :

\begin{lstlisting}
ipconfig
\end{lstlisting}

Puis lancer le client :

\begin{lstlisting}
cd ~/tetris_online/client/build
./tetris xx.xx.xx.xx
\end{lstlisting}

où \texttt{xx.xx.xx.xx} correspond à l'adresse IP Windows.

\begin{center}
\textbf{Exemple :} \texttt{./tetris 10.90.234.216}
\end{center}

\textbf{Attention :} ne jamais utiliser l'adresse IP WSL côté client.

% ========================
\section{Cas d'une exécution sous Linux natif}

Si le projet est exécuté sur une machine Linux native, WSL n'est pas nécessaire.

Dans ce cas :
\begin{itemize}
    \item Installer Raylib
    \item Compiler le projet
    \item Lancer le serveur et le client
\end{itemize}

Le client se connecte directement à l'adresse IP Linux du serveur :

\begin{lstlisting}
./tetris IP_DU_SERVEUR_LINUX
\end{lstlisting}

Aucune redirection de port n'est requise.

% ========================
\section{Important : copier-coller des commandes}

Il est déconseillé de copier-coller directement les commandes depuis le PDF
vers le terminal.

Certains caractères peuvent être mal copiés (espaces, tirets, antislashs),
ce qui provoque des erreurs difficiles à diagnostiquer.

Il est recommandé de :
\begin{itemize}
    \item coller la commande dans un LLM (ex : ChatGPT)
    \item vérifier et corriger la syntaxe
    \item puis exécuter la commande dans le terminal
\end{itemize}

% ========================
\section{Remarques finales}

\begin{itemize}
    \item Le serveur s'exécute sous WSL ou Linux natif
    \item Les clients utilisent l'adresse IP de la machine serveur
    \item Le port utilisé est \textbf{4242}
    \item Les commandes PowerShell doivent être exécutées en administrateur
    \item En cas de problème réseau, vérifier que le serveur écoute sur \texttt{0.0.0.0}
\end{itemize}

\end{document}
